%
%  untitled
%
%  Created by Johan Boissard [] on 2010-02-14.
%  Copyright (c) Johan Boissard. All rights reserved.
% hhh
\documentclass[a4paper]{article}


% Packages
\usepackage[french]{babel}
\usepackage[utf8]{inputenc}
\usepackage[T1]{fontenc}

\usepackage{amsmath,amssymb}
%allows to write on whole page
\usepackage{fullpage}


%package for page numbering and footer
\usepackage{fancyhdr}
%to get last page
\usepackage{lastpage} % \pageref{LastPage}
%allows inclusion of graphics
\usepackage{graphics}
\DeclareGraphicsExtensions{.pdf, .jpeg,.jpg}
%allows drawing
%ref:http://www.texample.net/
\usepackage{tikz}

%allows inclusion of url (hyperref is better than url) 
%ref: http://www.fauskes.net/nb/latextips/
\usepackage{hyperref}

%package for chemistry ie: \ce{(NH4)2SO4 -> NH4+ + 2SO4^2-} 
%ref:www.ctan.org/tex-archive/macros/latex/contrib/mhchem/mhchem.pdf
\usepackage[version=3]{mhchem}

%comments (begin{comment})
\usepackage{comment}

% Title
\title{Dirac}
\author{Johan Boissard}
\date\today

% Header


\pagestyle{fancy}
%delete current header & footer configuration
\fancyhf{} 
\fancyhead[EL]{Titel} %Kopfzeile links
%\fancyhead[C]{} %zentrierte Kopfzeile
%\fancyhead[LE,RO]{Name} %Kopfzeile rechts
%head separation line 
%\renewcommand{\headrulewidth}{0.4pt} 
%foot separation line
%\renewcommand{\footrulewidth}{0.4pt}
%page number
\fancyfoot[OL]{\thepage\ of \pageref{LastPage}}
%\fancyfoot[ER]{\thepage}

\begin{document}
\maketitle
\section{Fonction de Dirac}
\subsection{Propriétés}
\begin{eqnarray}
	\delta(t)=
	\begin{cases}
		\infty&t=0\\
		0&t\neq0
	\end{cases}
	\\
	\int_{-\infty}^{\infty}\delta(t)dt=1
\end{eqnarray}

\subsubsection{Transformée de Fourier}
\begin{eqnarray}
	\mathcal{F}\{\delta(t)\}(\omega)=1
	\\
	\mathcal{F}^{-1}\{\delta(t)\}(\omega)=2\pi
\end{eqnarray}

\subsubsection{Convolution et produit scalaire}
\begin{eqnarray}
	<f,\delta>=\int_{-\infty}^{\infty}f(x)\delta(x)dx=f(0)\int_{-\infty}^{\infty}\delta(x)dx= f(0)
	\\
	(f*\delta)(t)=f(t)
\end{eqnarray}
\section{Delta de Kronecker}
Equivalent en discret de la fonction de Dirac
\begin{eqnarray}
	\delta_{i,j}=
	\begin{cases}
		1&i=j\\
		0&i\neq j
	\end{cases}
\end{eqnarray}
Souvent utilisé de la façon suivante
\begin{eqnarray}
	\delta[n]=
	\begin{cases}
		1&n=0\\
		0&n\neq0
	\end{cases}
\end{eqnarray}
\subsection{Propriétés}
\begin{eqnarray}
	\sum_{j=-\infty}^{\infty}\delta_{i,j}a_i=a_i
\end{eqnarray}
\end{document}
